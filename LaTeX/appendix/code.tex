\subsection{DALL-E 3 API for Image Generation}

\textit{DALL-E} is now accessible through an API, allowing developers to integrate its capabilities directly into their applications.

Below is an example of how to use the DALL-E API in Python to generate and download an image. This code snippet demonstrates how to programmatically request DALL-E to create an image based on a specified prompt and then download the generated image to a local directory.


\begin{lstlisting}
import requests
import os

api_key = 'given-api-key-here'

def download_image(url, save_path, file_name):
    """
    Downloads an image from the specified URL and saves it to a local directory.
    """
    # Send a GET request to the URL
    response = requests.get(url)
    if response.status_code == 200:
        # Create the save directory if it doesn't exist
        os.makedirs(save_path, exist_ok=True)
        # Create the full path for the new file
        file_path = os.path.join(save_path, file_name)
        # Open the file and write the image content to it
        with open(file_path, 'wb') as file:
            file.write(response.content)
        print(f"Image downloaded successfully: {file_path}")
        return file_path
    else:
        # Raise an exception if there's an error in downloading
        raise Exception(f"Error downloading image: {response.status_code}")

def generate_image(prompt, model="dall-e-3", size="1024x1024", quality="standard", n=1, api_key = api_key):
    """
    Generates an image based on a text prompt using the DALL-E API.
    """
    # Set up the headers for the API request
    headers = {
        "Authorization": f"Bearer {api_key}",
        "Content-Type": "application/json"
    }

    # Define the data payload for the POST request
    data = {
        "model": model,
        "prompt": prompt,
        "size": size,
        "quality": quality,
        "n": n
    }

    # Make the POST request to the DALL-E API
    response = requests.post("https://api.openai.com/v1/images/generations", json=data, headers=headers)
    if response.status_code == 200:
        # Return the URL of the generated image
        return response.json()['data'][0]['url']
    else:
        # Raise an exception if there's an error in the API call
        raise Exception(f"Error in API call: {response.status_code} {response.text}")

def generate_and_download_image(prompt, save_path, file_name):
    """
    Combines the functions of generating an image and downloading it.
    """
    # Generate the image and retrieve the URL
    image_url = generate_image(prompt)
    # Download the image from the URL
    return download_image(image_url, save_path, file_name)

# Example usage
prompt = "a photorealistic image of a tsunami"
save_path = "downloads"
file_name = "tsunami.jpg"
generate_and_download_image(prompt, save_path, file_name)


\end{lstlisting}

\subsection{ImagineAPI.dev Midjourney API Image Generation}

Given the absence of an official \textit{Midjourney} API, third-party services such as \textit{ImagineAPI.dev} offer an alternative means of accessing \textit{Midjourney}'s image generation model. 

Below is an example of how to use their API in Python to generate and download the images. As \textit{Midjourney} generates four image variants for each call, this code randomly selects one of these four images for analysis and saves the other three as a reference.

\begin{lstlisting}
import http.client
import json
import time
import random
import os
import requests

def send_request(method, path, api_key, body=None):
    """Establishes a connection and sends a request to the server."""
    conn = http.client.HTTPSConnection("demo.imagineapi.dev")
    headers = {
        'Authorization': f'Bearer {api_key}',  # Authorisation header with API key
        'Content-Type': 'application/json'
    }
    conn.request(method, path, body=json.dumps(body) if body else None, headers=headers)
    response = conn.getresponse()
    data = json.loads(response.read().decode())
    conn.close()
    return data

def generate_image(prompt, api_key):
    """Generates an image based on the prompt and returns the image ID."""
    data = {"prompt": prompt}
    response_data = send_request('POST', '/items/images/', api_key, data)
    return response_data['data']['id']

def get_image_status(image_id, api_key):
    """Retrieves the status of the image generation."""
    response_data = send_request('GET', f"/items/images/{image_id}", api_key)
    return response_data['data']

def wait_for_image(image_id, api_key):
    """Waits for the image to be generated, checking the status every 5 seconds."""
    while True:
        image_data = get_image_status(image_id, api_key)
        if image_data['status'] in ['completed', 'failed']:
            return image_data
        else:
            print(f"Image generation in progress. Status: {image_data['status']}")
            time.sleep(5)

def download_images(image_data, download_path, selection_path):
    """
    Downloads the images from the provided image data.
    
    This function takes the list of image URLs from the 'upscaled_urls' field in the image data.
    It randomly selects one of these images to save in the 'selection_path' directory, 
    indicating it as the chosen image. The rest of the images are saved in the 'download_path' directory.

    Args:
        image_data: The data containing the image URLs.
        download_path: The path where the majority of images will be downloaded.
        selection_path: The path where the randomly selected image will be saved.
    """
    if not image_data or 'upscaled_urls' not in image_data:
        print("No image data provided for download.")
        return

    urls = image_data['upscaled_urls']
    os.makedirs(download_path, exist_ok=True)
    os.makedirs(selection_path, exist_ok=True)

    selected_image_url = random.choice(urls)
    for url in urls:
        file_name = url.split('/')[-1]
        file_path = os.path.join(selection_path if url == selected_image_url else download_path, file_name)

        response = requests.get(url)
        if response.status_code == 200:
            with open(file_path, 'wb') as file:
                file.write(response.content)
            print(f"Image downloaded successfully: {file_path}")
        else:
            print(f"Error downloading image from {url}: {response.status_code}")

# Example usage
api_key = 'given-api-key-here'
prompt = "a photorealistic image of a tsunami"

image_id = generate_image(prompt, api_key)
image_data = wait_for_image(image_id, api_key)

download_path = "downloads"
selection_path = "selection"
download_images(image_data, download_path, selection_path)

\end{lstlisting}



\subsection{Visual Attribute Detection}

This Python script is designed to analyse images and extract key visual attributes, specifically average colour, brightness, and saturation. The script performs the following steps:

\begin{enumerate}
    \item \textbf{Reading the Image}: The image is read in BGR (Blue, Green, Red) format using OpenCV.
    \item \textbf{Colour Conversion}: The image is then converted to RGB (Red, Green, Blue) format for colour analysis, and to HSV (Hue, Saturation, Value) format for brightness and saturation analysis.
    \item \textbf{Calculating Averages}: It computes the average colour in the RGB format, and the average brightness and saturation from the HSV format.
    \item \textbf{Output}: The script returns the calculated averages of colour, brightness, and saturation for further analysis.
\end{enumerate}

\begin{lstlisting}
import matplotlib.pyplot as plt
import numpy as np
import cv2

# Function to analyse an image and extract its visual attributes
def analyse_image(image_path):
    # Read the image in BGR format
    image = cv2.imread(image_path)
    
    # Convert the image to RGB format
    image_rgb = cv2.cvtColor(image, cv2.COLOR_BGR2RGB)
    
    # Convert the image to HSV format for brightness and saturation analysis
    image_hsv = cv2.cvtColor(image, cv2.COLOR_BGR2HSV)
    
    # Calculate the average color (in RGB)
    average_colour = np.mean(image_rgb, axis=(0, 1))
    
    # Calculate the average brightness (V channel of HSV)
    average_brightness = np.mean(image_hsv[:, :, 2])
    
    # Calculate the average saturation (S channel of HSV)
    average_saturation = np.mean(image_hsv[:, :, 1])
    
    # Display the image
    plt.imshow(image_rgb)
    plt.axis('off')
    plt.title('Analysed Image')
    plt.show()
    
    # Return the computed visual attributes
    return average_colour, average_brightness, average_saturation
\end{lstlisting}

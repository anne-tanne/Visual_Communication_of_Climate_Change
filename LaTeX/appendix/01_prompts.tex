\section{Prompts}
% Exclude subsections from the main ToC
\label{appendix:prompts}

\addtocontents{toc}{\protect\setcounter{tocdepth}{0}}

To explore the representation of extreme weather events by text-to-image models such as \textit{DALL-E} and \textit{Midjourney}, a set of prompts was developed. The approach starts with a general call to represent extreme weather events and then addresses specific types of events such as temperature extremes, heavy precipitation, floods, droughts, storms and hurricanes. The prompts also cover the impact of these events on human life, natural landscapes, infrastructure and cultural and social changes. Subsequent prompts address recovery and adaptation strategies, contributing factors and the presentation of future climate scenarios.

\subsection{General Prompt}
\begin{description}
\item \textbf{Prompt 1:} \textit{Depict an extreme weather event.}
\end{description}
\textbf{Relevance:} This prompt is designed to be deliberately broad, allowing the text-to-image models to generate images based on their inherent understanding and training data regarding extreme weather events.

\subsection{Specific Extreme Weather Events}
Following the general prompt, more specific prompts are introduced, each focusing on a particular type of extreme weather event. These prompts aim to dissect the models' portrayal of different weather phenomena and their impacts.

\begin{description}
\item \textbf{Prompt 2:} \textit{Depict temperature extremes.}
\item \textbf{Prompt 3:} \textit{Depict heavy precipitation}
\item \textbf{Prompt 4:} \textit{Depict pluvial floods}
\item \textbf{Prompt 5:} \textit{Depict river floods.}
\item \textbf{Prompt 6:} \textit{Depict droughts.}
\item \textbf{Prompt 7:} \textit{Depict extreme storms}
\item \textbf{Prompt 8:} Depict \textit{tropical cyclones}
\item \textbf{Prompt 9:} \textit{Depict compound weather events.}
\end{description}
\textbf{Relevance:} These prompts are derived from the types of extreme weather events as defined by the IPCC (see \ref{tab:extreme_weather_events}) They target specific weather phenomena to understand the models' capabilities in depicting each type of event.

\subsection{Impacts of Extreme Weather Events}
These calls aim to investigate how text-to-image models represent the impact of extreme weather events on different aspects of life and the environment (see chapter \ref{subsubsec:impacts-climate-change}).

\begin{description}
\item \textbf{Prompt 10:} \textit{Show the impact of extreme weather on human life.}
\item \textbf{Prompt 11:} \textit{Illustrate the effects of extreme weather on natural landscapes.}
\item \textbf{Prompt 12:} \textit{Visualise the infrastructural damage caused by extreme weather.}
\item \textbf{Prompt 13:} \textit{Depict the cultural and social changes resulting from extreme weather events.}
\item \textbf{Prompt 14:} \textit{Show a region affected by extreme weather.}
\end{description}
\textbf{Relevance:} These prompts aim to uncover the representation of the consequences of extreme weather events in the models, focusing on the human, natural and infrastructural impacts that play an important role in climate change studies.

\subsection{Action related to Extreme Weather Events}
This set of prompts is intended to generate imagery depicting outcomes or responses to extreme weather events.

\begin{description}
\item \textbf{Prompt 15:} \textit{Depict recovery efforts following extreme weather events.}
\item \textbf{Prompt 16:} \textit{Showcase adaptation strategies to extreme weather.}
\end{description}
\textbf{relevance:} The aim here is to investigate how text-to-image models visualise post-event scenarios and highlight the human capacity for resilience and adaptation in the face of climate-related challenges.

\subsection{Reasons for Extreme Weather Events}
These calls focus on the underlying causes or reasons for extreme weather events as represented by text-to-image models.

\begin{description}
\item \textbf{Prompt 17:} \textit{Depict the reasons contributing to extreme weather events.}
\item \textbf{Prompt 18:} \textit{Depict the human activities contributing to extreme weather events.}
\item \textbf{Prompt 19:} \textit{Show the natural factors leading to extreme weather events.}
\end{description}
\textbf{Relevance:} These prompts are intended to understand the text-to-image models' interpretation of the causes of extreme weather events.

\subsection{Future Climate and Weather Prompts}
This section includes prompts designed to explore how text-to-image models envision future weather scenarios and the long-term impacts of climate change.

\begin{description}
\item \textbf{Prompt 20:} \textit{"Depict weather patterns in a world where average global temperatures have risen by 2 degrees Celsius."}
\item \textbf{Prompt 21:} \textit{"Illustrate the changes in spring weather patterns due to global warming."}
        \begin{description}
            \item \textbf{Prompt 21b:} \textit{"Illustrate the changes in summer weather patterns due to global warming"}
            \item \textbf{Prompt 21c:}\textit{"Illustrate the changes in autumn weather patterns due to global warming"}
            \item \textbf{Prompt 21d:}\textit{"Illustrate the changes in winter weather patterns due to global warming"}
        \end{description}
\item \textbf{Prompt 22:} \textit{"Showcase the impact of sea-level rise in a coastal city in the future."}
\item \textbf{Prompt 23:} \textit{"Depict the future state of polar regions (if current warming trends continue)."}
\item \textbf{Prompt 24:} \textit{"Visualise future agricultural landscapes affected by changing climate conditions."}
\item \textbf{Prompt 25:} \textit{"Illustrate the transformation of urban areas in response to future extreme weather events."}
\item \textbf{Prompt 26:} \textit{"Depict human life adaptation in a future with frequent extreme weather events."}
\item \textbf{Prompt 27:} \textit{"Visualise future solutions to combat extreme weather challenges."}
\item \textbf{Prompt 28:} \textit{"Visualise the development of future solutions to combat extreme weather challenges."}
\end{description}

\textbf{Relevance:}
These prompts focus on the visualisation of possible future scenarios influenced by ongoing climate change. They aim to stimulate text-to-image models to generate images that reflect possible changes in weather patterns, environmental conditions and human adaptations. This research is critical to understanding the potential long-term impacts of climate change and the effectiveness of AI in visualising these future possibilities.

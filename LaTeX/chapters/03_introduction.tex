\section{Introduction}

In recent years, the discourse surrounding climate change has evolved from abstract predictions to an urgent reality marked by real-time observable impacts. Among the most visible and alarming manifestations of this global phenomenon is the increase in extreme weather events. These events, growing more frequent and severe \parencite{ipcc2023_wg1_11}, are altering ecosystems, impacting lives, and affecting economies globally \parencite[2460]{ipcc2023_wg2_16}. The Intergovernmental Panel on Climate Change (IPCC) has identified these changes as a direct consequence of climate change, highlighting the need for effective communication strategies to foster public understanding and engagement \parencite{ipcc2023_wg1_1}.

In the realm of climate change communication, visual representations play a crucial role. Research indicates that images not only shape emotions but also influence behaviours towards climate change \parencite{Leiserowitz2006}. Images enable a more rapid understanding of complex environmental risks \parencite{Epstein1994, Joffe2008} and catalyse the shift from passive observations to active participation \parencite{Keib2018}. Furthermore, images help to retain information better than text-only information \parencite{Coleman2009, Graber1990}, and provided readers have the same cultural references, images can overcome linguistic or geographical barriers when conveying information to an audience \parencite{Armfield2013}.

Given this understanding of the power of visual media, the advent of advanced text-to-image generation models like \textit{OpenAI}’s \textit{DALL-E} or \textit{Midjourney} introduces new opportunities and challenges in the field of climate communication. Text-to-image generation tools harness machine-learning algorithms to transform textual prompts into detailed visual representations \parencite{Zhang2023}. There are two sides to this: on the one hand, these tools offer novel means of representation and image extraction, and thus an innovative way of presenting nuanced impacts and imagined scenarios of climate change; on the other hand, they raise questions about the authenticity and reliability of visual media.

Recent examples of AI-generated content being used in journalism signify a broader trend in the media industry where the lines between AI-generated and human-generated content are becoming increasingly blurred \parencite{Henrich2023, Kim2023}. This trend is emphasised by an emerging discourse about the urgent need for comprehensive guidelines to navigate this new terrain \parencite{CouncilOfEurope2023, SwissPressCouncil2023}. 

Building on these developments, this paper delves into the emergent field of AI-augmented visual climate communication, particularly focusing on the depiction of extreme weather events. While the existing literature has largely focused on the impact of textual AI models like \textit{ChatGPT}, the role of AI in generating visual content, especially within the domain of scientific and climate communication, so far, has not been researched much. This paper explores this field by focussing on an analysis of outputs of extreme weather event imagery  from the \textit{DALL-E 3} and \textit{Midjourney} models.

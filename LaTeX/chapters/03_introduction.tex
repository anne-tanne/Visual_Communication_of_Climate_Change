\section{Introduction}

In recent years, climate change discourse has shifted from abstract predictions to an urgent reality with observable impacts. Among the most alarming manifestations is the increase in extreme weather events, which have intensified in frequency and severity \parencite{ipcc2023_wg1_11}. These events are disrupting ecosystems, affecting lives, and impacting economies globally \parencite[2460]{ipcc2023_wg2_16}. In this context, effective climate change communication is crucial \parencite{ipcc2023_wg1_1}. 

Visual representations play a vital role in climate change communication. Images shape emotions and influence behaviours towards climate change \parencite{Leiserowitz2006}. They aid in understanding complex environmental risks \parencite{Epstein1994, Joffe2008} and encourage active engagement \parencite{Keib2018}. Moreover, images are more memorable than text \parencite{Coleman2009, Graber1990} and can transcend language and geographical barriers if readers share cultural references \parencite{Armfield2013}.

Given this power of visual media, the emergence of advanced text-to-image models presents new prospects and challenges in climate communication. Text-to-image models harness machine-learning algorithms to transform textual prompts into visual representations \parencite{Zhang2023}, offering innovative ways to depict climate change impacts and (imagined) scenarios. However, they also raise concerns about the authenticity and reliability of visual media.

Recent instances of AI-generated content used in journalism \parencite{Henrich2023, Kim2023} indicate a growing trend towards its increased use in the field. This shift is also supported by a study by \textcite{Deloitte2023} which, though not exclusive to journalists, shows that 61\% of Swiss professionals using computers use generative technology. This highlights the importance of examining  the results of generative AI to assess its accuracy and justify its use in communications.

Hence, this paper investigates the use of text-to-image models, specifically \textit{DALL-E 3} and \textit{Midjourney v6}\footnote{hereinafter only referred to as \textit{DALL-E} and \textit{Midjourney}} in visualising extreme weather events and explores how these technologies represent such events.




